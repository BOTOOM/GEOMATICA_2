\chapter{Problema}
%%Este es el problema \cite{Bol_2012,Bol_2016}


\textbf{Problema:}
\\
Dificultad que tienen las personas para transportase por la ciudad mediante el uso de las bicibletas.
\\
\\
\section{Objetivos}

Diseñar y crear un sistema de infroamcio geografico (SIG), que permita a los ciudadanos que usan bicicletas para tranpportarse una facilidad al conocer las rutas que los llevaran a su destino, no solo las mas cortas, tabien se prioriza en la seguridad.
\\
\\
\textbf{Pregunta: ¿Cómo facilitar el transporte en cicla , dando rutas seguras y ofreciendo servicios en caso de algun inconveniente o calamidad?}
\\
\\
\section{Hipotesis}

Con un sistema de informacion geografico (SIG) dar al usuario el servicio de una ruta optima para el destino al que se quiera dirigir, priorizando en su reguridad, esto incluye, dar una ruta donde se usen la mayorparte de ciclovias, dar un pequeño informe de las rutas peligrosas en las cuales ocurren robos y/o accidentes.
\\
\\
Se dara no solo una ruta, ya que se dejara la ruto optima indicando que partes de esa ruta pueden ser peligrosas, y sus rutas alternas para evitar peligros, y en caso de algun inconveniente dar un listado de los talleres mas cercanos a los que puede ir.\\
\\
En caso de un accidente dar la ubicacion de los hopítales y lugares de ayuda mas cercanos y su ruta optima.
\\
\\
\vfill
\section{Justificación}

En la actualidad exiten muchos sistemas que proporcionan rutas optimas para el transporte , yasea en automovil, bicicleta o a caminando, pero no tiene en cuenta la seguridad de quien se tranporta; se quiere priorizar en la seguridad y ayuda al usuario, que tenga rutas seguras y alternas para su viaje y ayudas en caso de algun inconveniente 



        
