\chapter{Problema}
%%Este es el problema \cite{Bol_2012,Bol_2016}

\section{Introducción}

El crecimiento vehicular acelerado en las ciudades colombianas durante las últimas décadas ha hecho que la infraestructura vial sea insuficiente para movilizar eficientemente la población, causando principalmente: I) accidentalidad vial, II) sobre costos en el transporte, III) altos índices de contaminación atmosférica y acústica, y IV) pérdida de tiempo en los desplazamientos.\\
 
Es por ello que algunas personas han optado por usar sus bicicletas como medio de transporte para ir a sus sitios de interés (trabajo, estudio, otros), a pesar de los riesgos que puede acarrear esta decisión. 
Por ello este documento está enfocado al desarrollo de una aplicación de manejo de rutas y alertas en Colombia, específicamente En Bogotá D.C, una ciudad bastante amigable para este proyecto ya que cuenta con más de 120 km ce ciclo rutas; Esto debido a factores como son la congestión y desorganización en el sistema de transporte de la ciudad.\\

Teniendo en cuenta lo anterior se propone la elaboración de un SIG (Sistema de Información Geográfica) que permita a los bici-usuarios, mejorar su experiencia a la hora de salir a usar estas vías, además de ofrecerles información valiosa como lo son las rutas disponibles, puestos de reparación de bicicletas, puntos de hidratación, hospitales cercanos, bici parqueaderos, vías seguras e inseguras de la ciudad y restricciones que puedan tener las vías, entre otros.


\section{Objetivos}

\subsection{General}

Desarrollar un sistema que apoye en el control, el seguimiento de rutas y servicios enfocados a os bici-usuarios en la ciudad de Bogotá con el fin de mejorar la calidad de la movilidad y la seguridad de estos.

\subsection{Específicos}

\begin{itemize}
	\item Formar una red de usuarios para que conozcan la aplicación y así de esta manera crear la retroalimentación de la aplicación.
	\item Realizar la implementación de un SIG para la solución de problemas en el modelo de transporte de la bicicleta en la ciudad de Bogotá, problemas comunes como: 
	\begin{itemize}
		\item Calcular rutas.
		\item Indicadores de vías dañadas.
		\item Indicadores de sectores peligrosos.
		\item Indicadores de accidentes de tránsito.
		\item Indicadores de Centro médicos.
		\item Indicadores de Centros de atencion integral (CAI).
		\item Indicadores de talleres de reparacion y mantenimiento.
	\end{itemize}
\end{itemize}

\section{Productos a obtener}

En la finalización del proyecto se espera obtener una aplicación en equipos de escritorio o en equipos móviles, en donde se puede realizar la búsqueda y ubicación georreferenciada de las  ciclo vías de las cuales dispone nuestra ciudad Bogotá; incluiremos también puntos de hidratación, atención y reparación de bicicletas, calzadas de ciclo vías disponibles y habilitadas para su uso, indicador de congestión, vías en reparación, vías reportadas como peligrosas, centros de atención integral, centros medios y posibles rutas para llegar de un origen a un destino.

\clearpage
\section{Alcances y limitaciones}

Este trabajo de desarrollo abarca únicamente las calles y ubicaciones dentro de la ciudad de Bogotá teniendo en cuenta el mercado laboral en el que se encuentran y más específicamente la comunidad de personas para las cuales va destinada el desarrollo del aplicativo, que se inician por primera vez en la utilización de un SIG para el desplazamiento cotidiano. \\

Tiene como alcance sistematizar las operaciones de desplazamiento basado en un sistema de información geográfica, se busca que sea de fácil implementación y utilización por parte de los dispositivos tecnológicos en los cuales se descarguen, con miras al mejoramiento de la calidad e imagen de cada una de estas.







        
